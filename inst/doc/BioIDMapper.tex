%\VignetteIndexEntry{BioIDMapper Overview}
%\VignetteDepends{BioIDMapper}
%\VignetteKeywords{BioIDMapper}
%\VignetteKeywords{BioIDMapper}
%\VignetteKeywords{BioIDMapper}
%\VignettePackage{BioIDMapper}
\documentclass[a4paper]{article}

\newcommand{\Rfunction}[1]{{\texttt{#1}}}
\newcommand{\Robject}[1]{{\texttt{#1}}}
\newcommand{\Rpackage}[1]{{\textit{#1}}}
\newcommand{\Rclass}[1]{{\textit{#1}}}
\newcommand{\Rmethod}[1]{{\textit{#1}}}

\author{Xiaoyong Sun$^\dagger$$^\ddagger$\footnote{sunx1@iastate.edu}}

\usepackage{c:/Software/R-2.6.0/share/texmf/Sweave}
\begin{document}

\setkeys{Gin}{width=1\textwidth} 

\title{Quick Guide for BioIDMapper Package}
\maketitle
\begin{center}$^\dagger$Binformatics and Computational Biology Program, $^\ddagger$Department of Statistics \\ Iowa State University, Ames, Iowa 50010, USA
\end{center}

\tableofcontents
%%%%%%%%%%%%%%%%%%%%%%%%%%%%%%%%%%%%%%%%%%%%%%

\section{Introduction}
Many new databases aiming at genes and proteins are developed as more and more species are sequenced. It becomes tedious job about how to navigate among different data resources, map various IDs, and analyze separate biological knowledge.  Current popular databases include Entrez Gene, UniProt, Gene Ontology, EMBL, OMIM, PubMed, KEGG, etc. Based on NCBI, UniProt, KEGG and other web services,  BioIDMapper can facilitate mapping between different databases, integrate various ID systems and provide a full view from gene level, mRNA level and functional level regarding one specific ID. This package is based on NCBI and UniProt websites, utilizing two packages: XML and RCurl.

\section{BioID mapping table}
In this package, the following 59 BioIDs can be translated to each other. \newline \newline
\textit{Note:} \newline
(1)"GI number" shows in both NCBI and UniPort databases, and it serves
as bridge between two databases. \newline
(2) "Biokey number" is the "currency" connecting functions.\newline

\begin{center}
\begin{tabular}{ l l l}
\textbf{Biokey number}	& \textbf{BioIDs} &	\textbf{Sources} \\ 
1	& GI number	& NCBI \\
2	& Pubmed id &	NCBI \\
3	& GEO id & NCBI      \\
4	& OMIM id	& NCBI       \\
5	& SNP id	& NCBI       \\
6	& UniGene cluster id	& NCBI \\
7	& UniSTS id	& NCBI          \\
8	& Popset id	& NCBI           \\
9	& MMDB id	& NCBI              \\
10	& 3D SDI id &	NCBI          \\
11	& PSSM id	& NCBI            \\
12	& TAXID	& NCBI               \\
13	& Genome id	& NCBI            \\
14	& PubChem Compound id	& NCBI   \\
15	& PubChem Substance id	& NCBI  \\
16	& PubChem BioAssay id	& NCBI     \\
17	& NNNNNN	& Boundary              \\
18	& GI number	& UniProt             \\
19	& UniProtKB Accession	& UniProt   \\
20	& UniProtKB id	& UniProt        \\
21	& PIR Accession	& UniProt        \\
22	& Enzyme Commission	& UniProt    \\
23	& GO id	& UniProt                \\
24	& Entrez Gene id &	UniProt      \\
25	& EMBL id	& UniProt               \\
26	& ENSEMBL id	& UniProt          \\
27	& UniGene id	& UniProt          \\
28	& TAIR id	& UniProt              \\
29	& TIGR id	& UniProt              \\
30	& KEGG id	& UniProt              \\
\end{tabular}
\newpage
\begin{tabular}{ l l l}
\textbf{Biokey number}	& \textbf{BioIDs} &	\textbf{Sources} \\ 
31	& NCBI Taxon id	& UniProt        \\
32	& OMIM id	& UniProt              \\
33	& Ecogene id &	UniProt          \\
34	& Flybase id	& UniProt          \\
35	& GENEDB\_SPOMBE id	& UniProt    \\
36	& GERMONLINE id	& UniProt        \\
37	& GRAMENE id	& UniProt          \\
38	& HIV id	& UniProt              \\
39	& IPI	& UniProt                  \\
40	& PDB id	& UniProt              \\
41	& REBASE id	& UniProt            \\
42	& Refseq Accession &	UniProt    \\
43	& SGD id &	UniProt              \\
44	& TRANSFAC id	& UniProt          \\
45	& WORMPEP id &	UniProt          \\
46	& UniRef100 id &	UniProt        \\
47	& UniRef90 id &	UniProt          \\
48	& UniRef50 id	& UniProt          \\
49	& InterPro id &	UniProt          \\
50	& Medline id &	UniProt          \\
51	& PFAM id	& UniProt              \\
52	& PIRSF id &	UniProt            \\
53	& PRINTS id	& UniProt            \\
54	& PRODOM id	& UniProt            \\
55	& PROSITE id &	UniProt          \\
56	& PMID	& UniProt                \\
57	& SMART id	& UniProt            \\
58	& TAXGRPID	& UniProt            \\
59	& TIGRFAMs id & UniProt          \\
60	& TRANSFAC id &	UniProt          \\
\end{tabular}
\end{center}

\section{Function description} 
This package includes one standard mapping table displayed by bio.type()function, and three features: mapping, linking and data analysis. 
Mapping feature is implemented by bio.convert() function; linking feature involves bio.link() function, 
and data analysis is done by 2 functions: bio.sum() and bio.select().

\subsection{Retrieve mapping table}
\subsubsection{bio.type}
\begin{verbatim}
bio.type <-function(type2id)
\end{verbatim}
Show all Biokey numbers, and biological types that this package can handle. Presently 59 biological types are included in the package.
\newline
It takes one parameters: \newline
a. type2id \newline
Biokey number or BioIDs (biological types) from BioID mapping table; \newline
If no argument is used, return BioID mapping table for all biokey numbers;\newline
If argument is Biokey number from bio.type(), return the corresponding biological type;\newline 
If argument is biological type, return the corresponding Biokey number.\newline

\begin{verbatim}
>bio.type()

      Biokey number BioIDs                 Sources   
 [1,] "1"           "GI number"            "NCBI"    
 [2,] "2"           "Pubmed id"            "NCBI"    
 [3,] "3"           "GEO id"               "NCBI"    
 [4,] "4"           "OMIM id"              "NCBI"    
 [5,] "5"           "SNP id"               "NCBI"    
 [6,] "6"           "UniGene cluster id"   "NCBI"    
 [7,] "7"           "UniSTS id"            "NCBI"    
 [8,] "8"           "Popset id"            "NCBI"    
 [9,] "9"           "MMDB id"              "NCBI"    
[10,] "10"          "3D SDI id"            "NCBI" 
.....
\end{verbatim}
\begin{Schunk}
\begin{Sinput}
> library(BioIDMapper)
> bio.type(5)
\end{Sinput}
\begin{Soutput}
[1] "SNP id"
\end{Soutput}
\begin{Sinput}
> bio.type("SNP id")
\end{Sinput}
\begin{Soutput}
[1] 5
\end{Soutput}
\end{Schunk}
\subsection{Mapping feature}
\subsubsection{bio.convert}
\begin{verbatim}
bio.convert <-function(id_list, from, to)
\end{verbatim}
This is the main interface for mapping ids.
It takes three parameters: \newline
a. id\_list: id list you want to map; \newline 
b. from: Biokey number of source type; bio.type() will show all the Biokey numbers for biological types. \newline
c. to: Biokey number of destination type; bio.type() will show all the Biokey numbers for biological types.\newline

\begin{Schunk}
\begin{Sinput}
> data(glist)
> myMap <- bio.convert(glist, 1, 5)
> myMap[1:10, ]
\end{Sinput}
\begin{Soutput}
   GI number snp       
1  "200533"  "27242152"
2  "200533"  "27242151"
3  "200533"  "27242150"
4  "200533"  "27242149"
5  "200533"  "27242148"
6  "200533"  "27242147"
7  "200533"  "27242146"
8  "200533"  "27242145"
9  "200533"  "27242144"
10 "200533"  "27242143"
\end{Soutput}
\end{Schunk}

\subsection{Linking feature}
\subsubsection{bio.link}

\begin{verbatim}
bio.link <-function(id, to)
\end{verbatim}
This is the main interface for linking to external data sources. 
It will start web browser, and link that id to external data source.\linebreak[2]
It takes two parameters: \newline
a. id: id you want to link;  \newline
b. to: The corresponding Biokey number of external biology types you want to link; bio.type() will show all Biokey number for biological types. \newline
\textit{Note:} \newline 
"id" should match "to". For example, id: "27242148" is "SNP id"; to: 5 is Biokey number for "SNP id".

\begin{verbatim}
>bio.link("27242148", 5)
\end{verbatim}

\subsection{Data analysis feature}
This feature is to analyze result from mapping function: bio.convert().
\subsubsection{bio.sum}
\begin{verbatim}
	bio.sum <-function(result_matrix, start_idList, option)
\end{verbatim}
Summary the result after mapping. \linebreak[2]
It takes three parameters: \newline
a. result\_matrix: result matrix from bio.convert() function\newline
b. start\_idList: the orginial id list you want to map\newline
c. option: a logical value. If TRUE, all summary results are returned. If FALSE, only basic summary is returned. default value is FALSE \newline

\begin{Schunk}
\begin{Sinput}
> data(glist)
> myMap <- bio.convert(glist, 1, 5)
> bio.sum(myMap)
\end{Sinput}
\begin{Soutput}
------------------------------------------------------------------

MAPPING SCHEMA:
38 GI number  are mapped to  272 snp 
------------------------------------------------------------------

$`Summary for result`
             GI number snp
maping_Total        38 272
\end{Soutput}
\begin{Sinput}
> mySum <- bio.sum(myMap, glist, FALSE)
\end{Sinput}
\begin{Soutput}
------------------------------------------------------------------

MAPPING SCHEMA:
38 GI number  are mapped to  272 snp 

MAPPING PERCENTAGE:
100.00% GI number  are mapped to  snp 
------------------------------------------------------------------
\end{Soutput}
\end{Schunk}

\subsubsection{bio.select}
\begin{verbatim}
bio.select <-function(myid, result_matrix, colno)
\end{verbatim}
Show mapping result for one id.
It takes three parameters: \newline
a. myid: id you are interested\newline
b. result\_matrix: result matrix from bio.convert() function\newline
c. colno: the column number of result\_matrix that contains id you are interested.\newline

\begin{Schunk}
\begin{Sinput}
> data(glist)
> myMap <- bio.convert(glist, 1, 5)
> bio.select(myMap, 1, "41386735")
\end{Sinput}
\begin{Soutput}
    GI number  snp       
609 "41386735" "43102393"
610 "41386735" "43102392"
611 "41386735" "43102391"
612 "41386735" "43102390"
613 "41386735" "43102389"
614 "41386735" "43102388"
615 "41386735" "43102387"
616 "41386735" "43102386"
617 "41386735" "43102385"
618 "41386735" "41694775"
619 "41386735" "41694774"
620 "41386735" "41694773"
621 "41386735" "41694772"
622 "41386735" "41694771"
623 "41386735" "41694770"
624 "41386735" "41694769"
625 "41386735" "41694768"
626 "41386735" "41694767"
627 "41386735" "41694766"
628 "41386735" "41694765"
\end{Soutput}
\end{Schunk}

\section{Demonstration}

To illustrate how to use this package, three examples are used, including mapping within NCBI, UniProt respectively, and mapping between NCBI and UniProt, to show how to translate different biological ids .

\subsection{Examples for mapping within NCBI}
Let's assume that  you have 500 Genbank gi numbers, and you are interested in related snp ids. \newline
First, you can find Biokey number from bio.type() function,
\begin{verbatim}
> bio.type()

      Biokey number BioIDs                 Sources   
 [1,] "1"           "GI number"            "NCBI"    
 [2,] "2"           "Pubmed id"            "NCBI"    
 [3,] "3"           "GEO id"               "NCBI"    
 [4,] "4"           "OMIM id"              "NCBI"    
 [5,] "5"           "SNP id"               "NCBI"    
 [6,] "6"           "UniGene cluster id"   "NCBI"    
 [7,] "7"           "UniSTS id"            "NCBI"    
 [8,] "8"           "Popset id"            "NCBI"    
 [9,] "9"           "MMDB id"              "NCBI"    
[10,] "10"          "3D SDI id"            "NCBI" 
.....
\end{verbatim}
Second, you can use the bio.convert() to map from one type of id to the other type of id:
\begin{Schunk}
\begin{Sinput}
> data(glist)
> myMap <- bio.convert(glist, 1, 5)
> myMap[80:91, ]
\end{Sinput}
\begin{Soutput}
   GI number snp       
80 "200529"  "27242149"
81 "200529"  "27242148"
82 "200529"  "27242147"
83 "200529"  "27242146"
84 "200529"  "27242145"
85 "200529"  "27242144"
86 "200529"  "27242143"
87 "200529"  "27242142"
88 "200529"  "27242141"
89 "200529"  "27242140"
90 "200529"  "27242139"
91 "200529"  "27242138"
\end{Soutput}
\end{Schunk}
In addition, you can analyze the mapping result with bio.sum() function
\begin{verbatim}
>bio.sum(myMap)

          [,1]      [,2]
          "protein" "snp"
mappingNo "38"      "272"
\end{verbatim}
Also, you can select the id you are interested from result:
 \begin{verbatim}
>bio.select(myMap, 1, "200529")
\end{verbatim}
Finally, you can check the detailed information about snp with id: "27242138" using bio.link(), and it will give you more detailed information from web browser.
\begin{verbatim}
>bio.link("27242138", 5)
\end{verbatim}

\subsection{Examples for mapping within UniProt}

Assume that we have 10 UniProt Accession numbers, and let's find the related PDB ids.\newline
First, you need to find out related id number in BioIDMapper package using bio.type function:
\begin{verbatim}
> bio.type()
.....
[19,] "19"       "UniProtKB Accession"  "UniProt" 
[20,] "20"       "UniProtKB id"         "UniProt" 
[21,] "21"       "PIR Accession"        "UniProt" 
[22,] "22"       "Enzyme Commission"    "UniProt" 
[23,] "23"       "Go id"                "UniProt" 
[24,] "24"       "Entrez Gene id"       "UniProt" 
[25,] "25"       "EMBL id"              "UniProt" 
[26,] "26"       "ENSEMBL id"           "UniProt" 
[27,] "27"       "UniGene id"           "UniProt" 
[28,] "28"       "TAIR id"              "UniProt" 
[29,] "29"       "TIGR id"              "UniProt" 
[30,] "30"       "KEGG id"              "UniProt" 
[31,] "31"       "NCBI Taxon id"        "UniProt" 
[32,] "32"       "OMIM id"              "UniProt" 
[33,] "33"       "Ecogene id"           "UniProt" 
[34,] "34"       "Flybase id"           "UniProt" 
[35,] "35"       "GENEDB_SPOMBE id"     "UniProt" 
[36,] "36"       "GERMONLINE id"        "UniProt" 
[37,] "37"       "GRAMENE id"           "UniProt" 
[38,] "38"       "HIV id"               "UniProt" 
[39,] "39"       "IPI"                  "UniProt" 
[40,] "40"       "PDB id"               "UniProt"
.....
\end{verbatim}
\begin{Schunk}
\begin{Sinput}
> bio.type("PDB id")
\end{Sinput}
\begin{Soutput}
[1] 40
\end{Soutput}
\end{Schunk}
Second, you can use the bio.convert() to map from one type of id to the other type of id:
\begin{Schunk}
\begin{Sinput}
> data(ulist)
> myMap <- bio.convert(ulist, 19, 40)
> myMap[1:10, ]
\end{Sinput}
\begin{Soutput}
   ACC      PDB_ID           
1  "P04925" "1AG2"           
2  "P04925" "1XYX"           
3  "P04925" "1Y15"           
4  "P04925" "1Y16-PDBSUM_ID-"
5  "Q4FJQ7" "1E1G"           
6  "Q4FJQ7" "1E1J"           
7  "Q4FJQ7" "1E1P"           
8  "Q4FJQ7" "1E1S"           
9  "Q4FJQ7" "1E1U"           
10 "Q4FJQ7" "1E1W"           
\end{Soutput}
\end{Schunk}
You can also utilize bio.sum(), bio.select(), and bio.link() tools to check the related information.

\subsection{Examples for mapping between NCBI and UniProt }
You can do the mapping between NCBI and Uniprot exactly as before. Currently the bridge is "GI number" between the mapping of NCBI and Uniprot.  

If  you are interested in translating UniProt Accession Number to SNP id,

\begin{Schunk}
\begin{Sinput}
> data(ulist)
> myMap <- bio.convert(ulist, 19, 5)
> myMap[1:10, ]
\end{Sinput}
\begin{Soutput}
      ACC   P_GI      snp
1  P04925 200533 27242152
2  P04925 200533 27242151
3  P04925 200533 27242150
4  P04925 200533 27242149
5  P04925 200533 27242148
6  P04925 200533 27242147
7  P04925 200533 27242146
8  P04925 200533 27242145
9  P04925 200533 27242144
10 P04925 200533 27242143
\end{Soutput}
\end{Schunk}
\section{Case study}
To illustrate how to use these features, "chicken" data package from bioconductor.org is utilized for demonstration. The following R code enable user to map data from Entrez Gene id to UniProtKB Accession number (in the package, 24 represents Entrez Gene id, and 19 represents UniProt Accession Number).

\begin{verbatim}
# load "chicken" data package from Bioconductor
>library(chicken)
>xx <- as.list(chickenENTREZID)

# collect multiple entries for one probe and delete "NA" entries. 
>uxx<-unlist(xx)
>myList<-unique(as.matrix(uxx[!is.na(uxx)]))

# map the Entrez Gene id to UniProt Accession Number
>library(BioIDMapper)
>bio.convert(myList, 24, 19)->result
\end{verbatim}

Also, the result can be linked directly to the related database website for more detailed information about that specific ID.
\begin{verbatim}
>bio.link(result[2,2],19) 
\end{verbatim}
In addition, the data analysis module offers services to summarize mapping results. 
\begin{verbatim}
>bio.sum(result, myList, F)
\end{verbatim}

\end{document}




